%!TEX root = ../Thesis.tex

% 1.2 Defining the scope of your thesis
% One of the first tasks of a researcher is defining the scope of a study, i.e., its area (theme, field) and the amount of information to be included. Narrowing the scope of your thesis can be time-consuming. Paradoxically, the more you limit the scope, the more interesting it becomes. This is because a narrower scope lets you clarify the problem and study it at greater depth, whereas very broad research questions only allow a superficial treatment.

% The research question can be formulated as one main question with (a few) more specific sub-questions or in the form of a hypothesis that will be tested.

% Your research question will be your guide as your writing proceeds. If you are working independently, you are also free to modify it as you go along.

% How do you know that you have drafted a research question? Most importantly, a research question is something that can be answered. If not, you have probably come up with a theme or field, not a question.

% Some tips:

% Use interrogative words: how, why, which (factors/situations) etc.
% Some questions are closed and only invoke concrete/limited answers. Others will open up for discussions and different interpretations.
% Asking “What …?” is a more closed question than asking “How?” or “In what way?”
% Asking “Why” means you are investigating what causes of a phenomenon. Studying causality is methodologically demanding.
% Feel free to pose partially open questions that allow discussions of the overall theme, e.g., “In what way …?”; “How can we understand [a particular phenomenon]?”
% Try to condense your research question into one general question – and perhaps a few more specific sub-questions (two or three will usually suffice).

\section{Scope of the thesis}
In my perception, the scope of this thesis was greatly expanding over the time. I started with a simple question of "How the torque vectoring/load balancing can be implemented in an electric car?". This question, however, could be addressed in two ways, one would be by theoretical divagations, second by practical implementation details.

Since it was a first time when DTU is participating in Formula Student competition, I could not work on already working solution from previous years.
Therefore, the objective of the thesis was in great manner implementation oriented and difficulty moved towards all small design decisions, each base subsystem and cooperation with independent teams. So I would actually put the twist on the original question: "How to implement maintainable, performant and safe electric car control system in the manner that it would be additionally able to provide torque vectoring functionality?".

In the following chapters, I will try to explain my design decisions their advantages, drawbacks and how they build up a complete control system.