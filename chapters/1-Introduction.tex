%!TEX root = ../Thesis.tex

% Your introduction has two main purposes: 1) to give an overview of the main points of your thesis, and 2) to awaken the reader’s interest. It is recommended to rewrite the introduction one last time when the writing is done, to ensure that it connects well with your conclusion.

% Tip: For a nice, stylistic twist you can reuse a theme from the introduction in your conclusion. For example, you might present a particular scenario in one way in your introduction and then return to it in your conclusion from a different – richer or contrasting – perspective.

% The introduction should include:

% The background for your choice of theme
% A discussion of your research question or thesis statement
% A schematic outline of the remainder of your thesis
% The sections below discuss each of these elements in turn.


\chapter{Introduction}

An invention of the electric car has closely followed the time when the first electric motor has been built. The first construction is dated as early as the year 1837 with the golden age of electric cars in the 1920s. Since then the market was suddenly overtaken by the vehicles with internal combustion engines, however increasing environmental awareness and technical advancement lead us to reconcile usage of electric motors. 
In the last decade we could observe steady growth in hybrid and full electric constructions, so now a day the majority of car producers has in offer electric vehicles (EVs).

\todo{
    Source and figure 
    1820\textbackslash\_\_\_\_\_\_\_\_\_\_\_/1920\textbackslash\_\_\_\_\_\_\_\_\_\_/2020?
}

From the engineering point of view, it is an exciting new field which needs to be explored 


Now a day we are observing a great expand in electric cars industry. The idea of fully electric cars for mass usage, which been moved aside due to technical difficulties for over 100 years hits the market again. This movement stimulates research processes all around the world as the new architecture allows for optimisation which was impossible or not feasible for cars with a combustion engine.


Especially, an interesting configuration is whenever the vehicle is using two or more electric engines on per wheel basis. In this situation, a well known mechanical differential system can't longer be deployed resulting in reduced traction on the curves. To overcome this issue each motor has to be controlled independently and with precision by the onboard computer unit.

Project description:     
The goal of this project is to develop, implement and test the central control unit of an electric vehicle with special emphasis on electronic differential systems. Particularly, implementation of load sharing and torque vectoring. Project this however also covers the communication with motor drivers, BMS and all the other peripherals.

\todo{Used setup!}