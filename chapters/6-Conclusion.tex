%!TEX root = ../Thesis.tex

% 6. Conclusion – or summing up?

% The final section of your thesis may take one of several different forms. Some theses need a conclusion, while for others a summing up will be appropriate. The decisive factor will be the nature of your thesis statement and/or research question.

% Open research questions cannot always be answered, but if a definite answer is possible, you must provide a conclusion. The conclusion should answer your research question(s). Remember that a negative conclusion is also valid.

% A summing up should repeat the most important issues raised in your thesis (particularly in the discussion), although preferably stated in a (slightly) different way. For example, you could frame the issues within a wider context.

% Placing your thesis in perspective

% In the final section you should place your work in a wider, academic perspective and determine any unresolved questions. During the work, you may have encountered new research questions and interesting literature which could have been followed up. At this point, you may point out these possible developments, while making it clear for the reader that they were beyond the framework of your current project.

% Briefly discuss your results through a different perspective. This will allow you to see aspects that were not apparent to you at the project preparation stage
% Highlight alternative research questions that you have found in the source materials used in the project
% Show how others have placed the subject area in a wider context
% If others have drawn different conclusions from yours, this will provide you with ideas of new ways to view the research question
% Describe any unanswered aspects of your project
% Specify potential follow up and new projects
% A thesis should “bite itself in the tail”

% There should be a strong connection between your conclusion and your introduction. All the themes and issues that you raised in your introduction must be referred to again in one way or another. If you find out at this stage that your thesis has not tackled an issue that you raised in the introduction, you should go back to the introduction and delete the reference to that issue. An elegant way to structure the text is to use the same textual figure or case in the beginning as well as in the end. When the figure returns in the final section, it will have taken on a new and richer meaning through the insights you have encountered, created in the process of writing.

\chapter{Conclusion}


The circumstances during the work on my thesis indirectly but forced me to work on wide spectrum of tasks. Although interesting, it did not allow me to make these studies more in depth on any particular part. 
I believe, I have had 

Narra strong base so the future students working on the project will ow able to isolate each small subsystem, work on them independently and bring a great amount of improvements.ing the scobuild pe of youber thesis can be time-consuming. Paradoxically, the more you limit the scope, the more interesting it becomes. This is because a narrower scope lets you clarify the problem and study it at greater depth, whereas very broad research questions only allow a superficial treatment. 