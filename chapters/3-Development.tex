\chapter{Development}
In the first phase development process was greatly filled up with waiting. I joined the project in February when the project was still only on paper, teams were formed to perform certain tasks and we began ordering of necessary parts.

Having experience within similar systems I instantly realised that a National Instruments' controller we had in house would be a good choice for the main control unit. CompactRIO, family name of controller, consist variety of devices within fast and reliable real-time processor thigh up within FPGA (Field-Programmable Gate Array) all packaged intro rugged industrial form.

From the variety of options I have chosen model cRIO9033 which has great proportion between the performance and sizing. It comes with following relevant parameters:
\begin{table*}[H]
    \centering
    \begin{tabular}{r|c}
        Processor & Dual Core Intel Atom E3825 (1.33 GHz) \\
        RAM & 2GB (DDR3L) \\
        FPGA & Xilinx Kintex-7 7K160T \\
        Extension slots & 4
    \end{tabular}
    \caption{cRIO relevant parameters}
    \label{tab:cRIO_param}
\end{table*}

As a consequence the software needed to be written in LabVIEW (G - language). LabVIEW has strong position in industry being used some great project like Space X or CERN, moreover it has been widely adopted also in automotive industry by companies as:
% \begin{table*}[H]
%     \centering
%     \begin{tabular}{c|c|c|c|c|c}
%         Audi & Toyota & Nissan & Honda & Bosh & Continental  \\
%         Delphi & Visteon & Autoliv & DaimlerChrysler & Land Rover & Jaguar \\
%         BMW & Volkswagen & TRW & General Motors & Eaton Corp & Lear Corp \\
%         Saab & Ford & Siemens VDO & Breed Technologies & Fiat & Magneti Marelli \\
%         Ferrari & Skoda & Suzuki & Valeo & Volvo
%     \end{tabular}
%     \ref{LabVIEW_adoption}
%     \caption{LabVIEW adoption in automotive industry}
%     \label{tab:my_label}
% \end{table*} 
Although, being unfamiliar with its graphical programming interface at first, I was glad to have opportunity to learn it. 