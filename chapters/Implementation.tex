%!TEX root = ../Thesis.tex
\chapter{Implementation}
As previously mentioned the main control unit has be deployed on cRIO9033 - National Intruments' controller equipped with real-time processor and re-configurable FPGA unit. The reasoning for usage of this device was that it has more than sufficient computation power for the task and with additional modules provides support for basic CAN communication as well as high level CANOpen abstraction.


In my setup I have been using module NI9853 to directly receive and send a CAN messages and NI9881 for communication with the motor controllers based on CANOpen standard.

\section{Master control unit software implementation}
From the early beginning I have been developing the code of master control unit using event based programming paradigm. The reasoning of this is asynchronous, real-time nature of of such unit.

Most simple classification of subsystems can be based on communication protocol being used. 

\subsection{CAN integration}
To start with I had to implement a CAN library so it would be possible to make certain threads wait for message with predefined id, one from array of ids or just any message. 

To achieve it, I have a functionality deployed on FPGA (Field Programmable Gate Array) which simply translates input/output messages into the arrays of 8x unsigned integers. Read messages  are furthermore pushed into the FIFO register and interrupt on real-time processor is called.

Upon initiation of real-time programme additional process is spawned in the background which is responsible for communication with the FPGA. For sending a message it simply pushes data to FPGA and marks it as ready to send. When it comes to reading the process waits for interrupt from FPGA and acknowledges it after data read is finished. 

Then look up table is check for existence of conditional variable for given id, it is created if does not exist, and updated with new message. Additionally conditional variable containing id of last received message id is updated.

Based to this architecture I implemented functions which can wait for any message, one with certain id or one from ids set.

\subsection{Device operation states}
Although, during operation everything is event based the whole application itself can be considered a state machine where normal operation is just one of the states.

\todo{fig Init->boot-up seq->operational->exit}

\subsection{Operational state}
Implemented controller provides following functionality:
\begin{itemize}
    \item Log pedals, motor, pumps status and all occurred errors
    \item Updates motors target torque based on calculated electronic differential state (torque vectoring), power limitation and acceleration pedal position
    \item Control pumps power based on motors temperature and current target torque
    \item Subsystems connection loss procedures
    \item (Emergency) shutdown
    \item Read pedals, BMS, LV BMS and motors status.
\end{itemize}