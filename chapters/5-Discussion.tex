%!TEX root = ../Thesis.tex

% 5. Discussion
% In many thesis the discussion is the most important section. Make sure that you allocate enough time and space for a good discussion. This is your opportunity to show that you have understood the significance of your findings and that you are capable of applying theory in an independent manner.

% The discussion will consist of argumentation. In other words, you investigate a phenomenon from several different perspectives. To discuss means to question your findings, and to consider different interpretations. Here are a few examples of formulations that signal argumentation:

% On the one hand … and on the other …
% However …
% … it could also be argued that …
% … another possible explanation may be …

\chapter{Discussion}
As shown in previous chapter for current point in time all measurable characteristics closely follow expected ones, however, I would like to discuss the current limitations and how they could be overcome in future implementations.

To start with the main bottleneck of the system right now, it is the time needed to measure pedals position being higher than 5ms. It could be greatly reduced by changing a analogue/decimal conversion prescaler which in theory could result in shortening the time by 64 times (in cost of precision). Moreover, sampling and sending data could be pipe-lined so they would happen simultaneously. As the sampling time would not longer be a problem we would hit the limitation of CAN bus capacity which could be increased twice by changing to CAN adapter supporting full speed operation (current one supports only baud rate equal to 500kbit/s). This all could be done, however, the question is if it is feasible to do so or it will be just a computational overhead? Current control interval of $6,8ms$ is far away beyond humans perception \todo{cite} so the engines respond seems to be immediate. 

One could argue if the performance of differential system would be sufficient as mechanical counterpart we are used to has only latency due to inertia of the components. However, considering the fact that human being can move their hands only within maximum speed of about $~3m/s$ \cite{human_reaction_time} in $6.8ms$ steering wheel could be only moved by $~7.7 degrees$. Resulted target torque error should be within $2.7\%$ which in my perception can be considered neglectable.

What could be improved though is an usage of CAN bus and filtering of the signal. For example pedal sensor could use simple algorithm to not send unchanged position data and only send status message time to time to confirm that it is running. It would reduce computational efforts when not needed. 
Also simple analogue low pass filters could be used to cut of high frequency part of the electromagnetic interference.



When it comes to computational part if the car would be finally running certain behaviours can be adjusted. Here for example we can consider the dependency between acceleration pedal and cooling pumps power. I designed this control work well, however, it can be further optimised taking into account pumps performance curve.


If we would think about the differential system, current implementation although working, is simple and based on assumptions which might not always be true. Especially it does not take into account possibility of wheels skidding on the road. It should still outperform basic mechanical solution as it will not increase torque on sliding wheel. However, if more sensor would have been used it is possible to further limit torque on sliding wheel, keeping it in right skidding momentum for the best grip and therefore helping to keep steering under control.

Moreover, if more pivot sensors would be used, implementing additional assistance systems like anti-lock braking or "enhanced traction" seems like really straight forward tasks within current setup. 