%!TEX root = ../Thesis.tex

% Your introduction has two main purposes: 1) to give an overview of the main points of your thesis, and 2) to awaken the reader’s interest. It is recommended to rewrite the introduction one last time when the writing is done, to ensure that it connects well with your conclusion.

% Tip: For a nice, stylistic twist you can reuse a theme from the introduction in your conclusion. For example, you might present a particular scenario in one way in your introduction, and then return to it in your conclusion from a different – richer or contrasting – perspective.

% The introduction should include:

% The background for your choice of theme
% A discussion of your research question or thesis statement
% A schematic outline of the remainder of your thesis
% The sections below discuss each of these elements in turn.


\chapter{Introduction}

Invention of electric car has closely followed the time when the first electric motor has been build. The first construction is dated as early as year 1837 with the golden age of electric cars in 1920s. Since then the market was suddenly overtaken by the vehicles with internal combustion engines, however increasing environmental awareness and technical advancement lead us to reconcile usage of electric motors. 
In the last decade we could observe steady grow in hybrid and full electric constructions, so now a day majority of car producers has in offer electric vehicles (EVs).

\todo{
    Source and figure 
    1820\textbackslash\_\_\_\_\_\_\_\_\_\_\_/1920\textbackslash\_\_\_\_\_\_\_\_\_\_/2020?
}

From the engineering point of view it is exciting new field which needs to be explored 


Now a day we are observing a great expand in electric cars industry. The idea of fully electric cars for mass usage, which been moved aside due to technical difficulties for over 100 years hits the market again. This movement stimulates research processes all around the world as the new architecture allows for optimisation which was impossible or not feasible for cars with a combustion engine.


Especially, an interesting configuration is whenever the vehicle is using two or more electric engines on per wheel base. In this situation well known mechanical differential system cannot longer be deployed resulting in reduced traction on the curves. To overcome this issue each motor has to be controlled independently and with precision by the onboard computer unit.

Project description: 	
The goal of this project is to develop, implement and test central control unit of an electric vehicle with special emphasis on electronic differential systems. Particularly, implementation of load sharing and torque vectoring. Project this however also covers the communication with motor drivers, BMS and all the other peripherals.




% 1.3 Outline
% The outline gives an overview of the main points of your thesis. It clarifies the structure of your thesis and helps you find the correct focus for your work. The outline can also be used in supervision sessions, especially in the beginning. You might find that you need to restructure your thesis. Working on your outline can then be a good way of making sense of the necessary changes. A good outline shows how the different parts relate to each other, and is a useful guide for the reader.

% It often makes sense to put the outline at the end of the introduction, but this rule is not set in stone. Use discretion: What is most helpful for the reader? The information should come at the right point – not too early and not too late.

\section{Outline}
This thesis consist of \todo{how many} chapters. It starts with general introduction you are reading now, which outlines general topic of the work being done, hopefully giving pleasant to read opening. 
In the next chapter I will go into more details about currently used solutions in electric as well as cars with internal combustion engines. Knowing current state of art and being aware of the limitations I have written theoretical considerations are base for implementation part as well as discuss the perfect case used to validate further implementation.
Chapter 4 provides comprehensive description on my design choices, used setup and implementation itself. This part is directly followed by information of how I did test my system and how well it performs against my theoretical considerations.
Then I let myself describe directions for the future implementation which I hope will come in handy for future teams taking care of this project.
Whole text is closed with a few closing comments from me, retrospection on the whole process of building electric racing car and my part in it.

\todo{Used set up!}