%!TEX root = ../Thesis.tex

% 1.1 Background
% The background sets the general tone for your thesis. It should make a good impression and convince the reader why the theme is important and your approach relevant. Even so, it should be no longer than necessary.

% What is considered a relevant background depends on your field and its traditions. Background information might be historical in nature, or it might refer to previous research or practical considerations. You can also focus on a specific text, thinker or problem.

% Academic writing often means having a discussion with yourself (or some imagined opponent). To open your discussion, there are several options available. You may, for example:

% refer to a contemporary event
% outline a specific problem; a case study or an example
% review the relevant research/literature to demonstrate the need for this particular type of research
% If it is common in your discipline to reflect upon your experiences as a practitioner, this is the place to present them. In the remainder of your thesis, this kind of information should be avoided, particularly if it has not been collected systematically.

% Tip: Do not spend too much time on your background and opening remarks before you have gotten started with the main text.





\section{Background}



% \section{CAN}
% \subsection{CANOpen}
% \section{Steering}
% \section{Torque vectoring / load sharing}
% \chapter{Existing solutions/ compare to IC vehicles}
% Literature review on recent developments within the topic and techniques used in combustion engine cars