%!TEX root = ../Thesis.tex

% 1.3 Outline
% The outline gives an overview of the main points of your thesis. It clarifies the structure of your thesis and helps you find the correct focus for your work. The outline can also be used in supervision sessions, especially in the beginning. You might find that you need to restructure your thesis. Working on your outline can then be a good way of making sense of the necessary changes. A good outline shows how the different parts relate to each other, and is a useful guide for the reader.

\section{Outline}
This thesis consist of \todo{how many} chapters. It starts with general introduction you are reading now, which outlines general topic of the work being done, hopefully giving pleasant to read opening. 
In the next chapter I will go into more details about currently used solutions in electric as well as cars with internal combustion engines. Knowing current state of art and being aware of the limitations I have written theoretical considerations are base for implementation part as well as discuss the perfect case used to validate further implementation.
Chapter 4 provides comprehensive description on my design choices, used setup and implementation itself. This part is directly followed by information of how I did test my system and how well it performs against my theoretical considerations.
Then I let myself describe directions for the future implementation which I hope will come in handy for future teams taking care of this project.
Whole text is closed with a few closing comments from me, retrospection on the whole process of building electric racing car and my part in it.