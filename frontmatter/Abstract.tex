%!TEX root = ../Thesis.tex


% Summary and foreword
% Most readers will turn first to the summary (or abstract). Use it as an opportunity to spur the reader’s interest. The summary should highlight the main points from your work, especially the thesis statement, methods (if applicable), findings and conclusion. However, the summary does not need to cover every aspect of your work. The main objective is to give the reader a good idea of what the thesis is about.

% The summary should be completed towards the end; when you are able to overview your project as a whole. It is nevertheless a good idea to work on a draft continuously. Writing a good summary can be difficult, since it should only include the most important points of your work. But this is also why working on your summary can be so useful – it forces you to identify the key elements of your writing project.

% There are usually no formal requirements for forewords, but it is common practice to thank your supervisors, informants, and others who have helped and supported you. If you have received any grants or research residencies, you should also acknowledge these.

% Note: Shorter assignments do not require abstracts and forewords. 


\chapter{Abstract}

In the following thesis author provides comprehensive description of process, decision making and final implementation of master control unit for the electric racing car. The workload covered is a patronate over design of intelligent sensors/controllers, implementing event based CAN library for real time system, control motor controllers over the CANOpen protocol, communication with a battery management system, logging system state and events, implementing safe and reliable boot/shutdown sequence and finally on the top of it controlling controlling the car under normal operation including adaptive cooling pumps control, torque vectoring as replacement for mechanical differential and regenerative braking.